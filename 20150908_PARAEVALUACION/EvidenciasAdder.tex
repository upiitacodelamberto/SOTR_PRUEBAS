\documentclass[12pt]{article}
%Este documento pretende ser un programa para calcular 
%la calificación de un discente a medida que el autor 
%va editando el mismo y se buscará generar una tabla 
%que muestre las calificaciones de los discentes.
\usepackage[spanish]{babel}
\usepackage[utf8]{inputenc}

\def\Centrar#1{\begin{center}#1\end{center}}
\def\GRUPO{3MV7}
%Se utilizaran variables contador de \TeX
\newcount\NumDEvid
\NumDEvid=-1
\def\Evidencia#1{
{\global\advance\NumDEvid by 1}
{\large\bf Evidencia \the\NumDEvid: #1}
}

\newcount\ChavezCortesVictorEvidCount
\ChavezCortesVictorEvidCount=0

%@param #1: puede ser una observacion o una calificación
\def\EvidenciaChavezCortes#1{
\global\advance\ChavezCortesVictorEvidCount by 1
Chávez Cortes Víctor&#1\\\hline
}

\newcount\AntonioCoutinioDanielEvidCount
\AntonioCoutinioDanielEvidCount=0
\def\EvidenciaAntonioCoutinio#1{
\global\advance\AntonioCoutinioDanielEvidCount by 1
Antonio Couti\~no Daniel De Jesús&#1\\\hline
}

\newcount\RomoGarciaMiguelEvidCount
\RomoGarciaMiguelEvidCount=0
\def\EvidenciaRomoGarcia#1{
\global\advance\RomoGarciaMiguelEvidCount by 1
Romo García Miguel Angel&#1\\\hline
}

\newcount\CadenaDiazDanielEvidCount
\CadenaDiazDanielEvidCount=0
\def\EvidenciaCadenaDiaz#1{
\global\advance\CadenaDiazDanielEvidCount by 1
Cadena Díaz Daniel&#1\\\hline
}

\newcount\GijonLeyvaErickEvidCount
\GijonLeyvaErickEvidCount=0
\def\EvidenciaGijonLeyva#1{
\global\advance\GijonLeyvaErickEvidCount by 1
Gijón Leyva Erick Servando&#1\\\hline
}

\newcount\SolanoJuarezJoseEvidCount
\SolanoJuarezJoseEvidCount=0
\def\EvidenciaSolanoJuarez#1{
\global\advance\SolanoJuarezJoseEvidCount by 1
Solano Juárez José Santiago&#1\\\hline
}

\newcount\FrancoGarciaOswaldoEvidCount
\FrancoGarciaOswaldoEvidCount=0
\def\EvidenciaFrancoGarcia#1{
\global\advance\FrancoGarciaOswaldoEvidCount by 1
Franco García Oswaldo Ignacio&#1\\\hline
}

\newcount\VazquezHernandezJorgeEvidCount
\VazquezHernandezJorgeEvidCount=0
\def\EvidenciaVazquezHernandez#1{
\global\advance\VazquezHernandezJorgeEvidCount by 1
Vazquez Hernández Jorge&#1\\\hline
}

\newcount\LeonJuanJavierEvidCount
\LeonJuanJavierEvidCount=0
\def\EvidenciaLeonJuan#1{
\global\advance\LeonJuanJavierEvidCount by 1
León Juan Javier&#1\\\hline
}
%DiazDuran
\newcount\DiazDuranOmarEvidCount
\DiazDuranOmarEvidCount=0
\def\EvidenciaDiazDuran#1{
\global\advance\DiazDuranOmarEvidCount by 1
Díaz Duran Omar&#1\\\hline
}

\newcount\OrtizTorresErickEvidCount
\OrtizTorresErickEvidCount=0
\def\EvidenciaOrtizTorres#1{
\global\advance\OrtizTorresErickEvidCount by 1
Ortiz Torres Erick&#1\\\hline
}

\newcount\ParraHernandezVictorEvidCount
\ParraHernandezVictorEvidCount=0
\def\EvidenciaParraHernandez#1{
\global\advance\ParraHernandezVictorEvidCount by 1
Parra Hernández Víctor Manuel&#1\\\hline
}

\begin{document}
%\hspace{\fill}México D.F. \today\\
\hspace{\fill}México D.F. Martes 8 de septiembre de 2015\\
%\vspace{.25in}
\vskip.25in
\noindent{\large\bf \GRUPO\\
%{\large\bf \GRUPO\\
SISTEMAS OPERATIVOS EN TIEMPO REAL}\\
%México D.F. \today\par
\vskip.25in
\noindent\Evidencia{Macros $\_$CRTIMP y $\_\_$cdecl}\par
Se pidió a los discentes buscar el signif\/icado de las 
macros $\_$CRTIMP y $\_\_$cdecl. También se planteó el 
primer ejercicio reto del curso: Se pide encontrar una 
manera (usando directivas del preprocesador) de contar 
cuántas veces se incluye un archivo a sí mismo antes de 
provocar que el preprocesador aborte.\par
Evidencias \the\NumDEvid\ recibidas (en tiempo y forma):\par
\Centrar{
\begin{tabular}{|l|p{2in}|}\hline
\EvidenciaAntonioCoutinio{$\_\_$cdecl??}
\EvidenciaCadenaDiaz{\#define $\_$CRTIMP   $\_$declspec(dllimport)}
\EvidenciaChavezCortes{$\_\_$WIN32$\_\_$??}
\EvidenciaDiazDuran{}
\EvidenciaGijonLeyva{$\_$WIN32 el más usual}
\EvidenciaFrancoGarcia{$\_$WIN32??}
\EvidenciaLeonJuan{}
\EvidenciaOrtizTorres{}
\EvidenciaParraHernandez{}
\EvidenciaRomoGarcia{$\_$WIN16 (para 16 bits),  $\_$WIN32 
(para 16?? y 64 bits), 
$\_\_$WIN64$\_\_$ (para 64 bits), 
$\_$WINDOWS$\_$ (para Watcom C/C++)}
\EvidenciaSolanoJuarez{$\approx$}
\EvidenciaVazquezHernandez{}
\end{tabular}
}
Ahora, continuando con el trabajo de {\it literate programming}. 
Se simulará que algunos estudiantes, entregaron la siguiente 
evidencia, es decir, lo que sería la evidencia 1.\par
\eject\noindent\Evidencia{Programa que indica nombre del día de la 
semana}\par
\Centrar{
\begin{tabular}{|l|p{2in}|}\hline
\EvidenciaAntonioCoutinio{Evidencia \the\NumDEvid\ simulada:
\the\AntonioCoutinioDanielEvidCount}
\end{tabular}
}
\end{document}